%
%  Nature Neuroscience Template - Root Document File
%
%  Author: Joe Monaco <jmonaco@jhu.edu> 
%  Created: Jan 23, 2011.
%  Last updated: September 25, 2015.
%
%  This work is licensed under the Creative Commons Attribution 4.0
%  International License. To view a copy of this license, visit
%  http://creativecommons.org/licenses/by/4.0/.
%
\documentclass[12pt]{article}

\usepackage[utf8]{inputenc}
\usepackage{graphicx}
\usepackage{amssymb,amsmath}
\usepackage{lipsum}
\usepackage{sectsty}
\usepackage{fullpage}
\usepackage[pagewise]{lineno}
\usepackage{acronym}
\usepackage[hidelinks]{hyperref}

% Nature citation format
\usepackage[nospace]{cite}
\usepackage{citesupernumber}
\usepackage{numberlabel}  % nature.sty is deprecated

% Customizing captions and float behavior
\usepackage[format=plain,labelformat=empty,font=small]{caption}
\usepackage{placeins}

% For creating nice looking tables
\usepackage{booktabs}
\usepackage{multirow}

% Choose your font: uncomment one of the below font packages or add your own
% \usepackage{pslatex} % for TNR
\usepackage{palatino} % for Palatino
% \usepackage{fontspec} % forces XeLaTeX engine, use \setmainfont (etc) below
% \setmainfont[Mapping=tex-text,Ligatures=TeX]{Adobe Caslon Pro} % choose any system font
% \setmainfont[Mapping=tex-text]{Hoefler Text} % ...such as Hoefler Text

% Set the manuscript title(s): full title is in titletext.tex and a shorter (4-5
% words) running title can also be provided below
\renewcommand{\title}{%!TEX root = main.tex
Lorem Ipsum Dolor sit Amet, New Words Consectetur Adipiscing Elit: Donec a Diam Lectus

% Title ideas:
%

}
\newcommand{\runningtitle}{Lorem Ipsum Dolor}

% Define the author lists, affiliations, and corresponding author information
\newcommand{\authorlistplain}{
  First~Author, Second~Author, Corresponding~Author
}
\newcommand{\authorlistannotated}{
  First~Author$^{1}$, 
  Second~Author$^{2}$, 
  Corresponding~Author$^{3\ast}$
}
\newcommand{\authoraffiliations}{
  $^1$Lorem Ipsum Institute, 
      Dolor University,
      Springfield, MO, USA; \\
  $^2$Biomedical Engineering Dept.,
      Lectus University,
      Springfield, MA, USA; \\
  $^3$Dept. of Neuroscience, 
      Adipiscing Elit University,
      Springfield, MI, USA; \\
}
\newcommand{\correspondinglastname}{Author}
\newcommand{\correspondingaddress}{
  Dr. Corresponding Author \\
  Adipiscing Elit University \\
  Department of Neuroscience \\
  Springfield, MI 54545, USA \\ 
  cauthor@adi.edu \\
}

% To render a selective list of manuscript sections, provide an includeonly command 
% with a comma-separated list of sections to include. (Comment out all includeonly 
% commands to render everything.) The available sections are any of the included 
% files in the document: title, abstract, text, endnotes, legends, methods, 
% figures, supp/legends, supp/guide, supp/figures, and supp/tables. Note that the
% presence of includeonly screws up latexdiff even when commented out, which is why
% the command names are obfuscated (and the stars should be removed when uncommenting.)
%
% \incl***udeonly{text}
% \incl***udeonly{legends,figures}
% \incl***udeonly{methods}
% \incl***udeonly{supp/guide}
% \incl***udeonly{supp/figures,supp/tables}

% To exclude the main figures from the manuscript file, uncomment this line:
% \def \excludefigures {}

% To exclude the references from the manuscript file, uncomment this line:
% \def \excluderefs {}

% Annotation and inline comment command(s): change the author=XXX parameters and add
% extra commands for additional authors to comment. Add `disable' to the package
% parameters to turn off all annotations
\usepackage[disable,shadow,linecolor=black,textsize=small]{todonotes}
\newcommand{\note}[1]{\todo[author=JDM,color=orange!40,linecolor=black]{#1}}
\newcommand{\comment}[1]{\todo[inline,author=JDM,color=orange!40,linecolor=black]{#1}}
% \newcommand{\xyznote}[1]{\todo[author=XYZ,color=green!30,linecolor=black]{#1}}

% Set case of figure panel labels
\newcommand{\panelcase}{\lowercase}
% \newcommand{\panelcase}{\uppercase}

% Reference commands - Figures (call like \[t]*fig{<key>}{<panel_string>} or
% \*figs{<key1>}{<key2>}). Keys reference figures based on figures/Figures_<key>.png 
% image files that are brought in via \mainfigure commands in figures.tex.
\newcommand{\fig}[2]{Fig.~\ref{fig:#1}\panelcase{#2}}
\newcommand{\figs}[2]{Figs.~\ref{fig:#1} and \ref{fig:#2}}
\newcommand{\tfig}[2]{Figure~\ref{fig:#1}\panelcase{#2}}
% Supplementary figures
\newcommand{\suppfig}[2]{Supplementary Fig.~\ref{suppfig:#1}\panelcase{#2}}
\newcommand{\suppfigs}[2]{Supplementary Figs.~\ref{suppfig:#1} and \ref{suppfig:#2}}
\newcommand{\tsuppfig}[2]{Supplementary Figure~\ref{suppfig:#1}\panelcase{#2}}
% Supplementary tables
\newcommand{\supptable}[1]{Supplementary Table~\ref{tab:#1}}
% Equations
\newcommand{\eqn}[1]{equation~(\ref{eq:#1})}
\newcommand{\eqns}[2]{equations~(\ref{eq:#1}) and (\ref{eq:#2})}

% Figure and caption display commands:
% NOTE: Main figures should live at ./figures/Figures_<name>.png and
% supplementary figures should live at ./figures/SuppFigures_<name>.png.
% Then use the above reference commands with <name> as the figure key.
% Figure images should be single-column (89mm) or double-column (183mm) wide.
\newcommand{\panel}[1]{\panelcase{\textbf{#1}}}
\newcommand{\customfigure}[4]{
  \begin{figure}[hp]
    \centering
    \includegraphics[scale=1]{figures/#1Figures_#3.png}
    \caption{#4\label{#2fig:#3}}
  \end{figure}
  \vfill
  \clearpage
}
\newcommand{\mainfigure}[1]{
  \customfigure{}{}{#1}{\textbf{\tfig{#1}{}}}
}
\newcommand{\maincaption}[3]{
  {\bf\tfig{#1}{}} $|$ \textbf{#2.} #3
}
\newcommand{\suppcaption}[3]{
  {\bf\tsuppfig{#1}{} $|$} \textbf{#2.} #3
}
\newcommand{\suppfigurewithcaption}[3]{
  \customfigure{Supp}{supp}{#1}{\suppcaption{#1}{#2}{#3}}
}

% Set formatting for section headings
\allsectionsfont{\bfseries}
\sectionfont{\centering\Large}
\subsectionfont{\large}
\paragraphfont{\hspace*{1.5em}}

% Paragraph and line spacing
\setlength{\parindent}{0pt}
\setlength{\parskip}{4ex plus 1ex minus 0.4ex}
\frenchspacing
\raggedright

% Page setup: margins, footers, etc.
\setlength{\hoffset}{0pt}
\setlength{\textwidth}{6.5in}
\setlength{\topmargin}{0pt}
\setlength{\headheight}{0pt}
\setlength{\headsep}{0pt}
\setlength{\voffset}{0pt}
\setlength{\footskip}{5ex}
\setlength{\textheight}{9in}

% Comment this line to enable line numbering
\renewcommand{\linenumbers}{{}}

% Maximize orphan/widow penalties
\widowpenalty=10000
\clubpenalty=10000

\begin{document}

  %!TEX root = main.tex

\begin{titlepage}
  \vspace*{\stretch{1}}

  \begin{flushright}
      \emph{RUNNING TITLE:}\quad\runningtitle
  \end{flushright}

  \vspace{1cm}

  \begin{center}  
    {
      \fontseries{bx}
      \fontsize{16pt}{1.5\baselineskip}
      \selectfont
      \title
    }
  
    \vspace{1.5cm}
  
    {
      \normalsize
      \authorlistannotated
    }

    \vspace{1cm}
  
    {
      \footnotesize
      \setlength{\baselineskip}{1.6em}
      \authoraffiliations
    }
  \end{center}

  \vspace{1.5cm}

  \begin{flushleft}
    \small
    $^\mathbf{\ast}$\textbf{Correspondence:} \\[0.4em]
    \correspondingaddress
  \end{flushleft}
  
  \vspace*{\stretch{4}}
\end{titlepage}



  \setlength{\baselineskip}{2.6em}
  
  {
    \linenumbers
    \thispagestyle{empty}
    %!TEX root = main.tex

\section*{Abstract}

Lorem ipsum dolor sit amet, consectetur adipiscing elit. Donec a diam lectus. Sed sit amet ipsum mauris. Maecenas
congue ligula ac quam viverra nec consectetur ante hendrerit. Donec et mollis dolor. Praesent et diam eget libero
egestas mattis sit amet vitae augue. Nam tincidunt congue enim, ut porta lorem lacinia consectetur. Donec ut libero sed
arcu vehicula ultricies a non tortor. Lorem ipsum dolor sit amet, consectetur adipiscing elit. Aenean ut gravida lorem.
Ut turpis felis, pulvinar a semper sed, adipiscing id dolor. Pellentesque auctor nisi id magna consequat sagittis.
Curabitur dapibus enim sit amet elit pharetra tincidunt feugiat nisl imperdiet. Ut convallis libero in urna ultrices
accumsan. Donec sed odio eros. Donec viverra mi quis quam pulvinar at malesuada arcu rhoncus. Cum sociis natoque
penatibus et magnis dis parturient montes, nascetur ridiculus mus. In rutrum accumsan ultricies. Mauris vitae nisi at
sem facilisis semper ac in est. 
  }

  {
    \setcounter{page}{3}
    \linenumbers
    %!TEX root = main.tex

\section*{Text}

Lorem ipsum dolor sit amet, consectetur adipiscing elit. Donec a diam lectus. Sed sit amet ipsum mauris. Maecenas
congue ligula ac quam viverra nec consectetur ante hendrerit. Donec et mollis dolor. Praesent et diam eget libero
egestas mattis sit amet vitae augue. Nam tincidunt congue enim, ut porta lorem lacinia consectetur\cite{OKeeNade78a}.
Donec ut libero sed arcu vehicula ultricies a non tortor. Lorem ipsum dolor sit amet, consectetur adipiscing elit.
Aenean ut gravida lorem. There is something fishy going on here. Ut turpis felis, pulvinar a semper sed, adipiscing id
dolor. Pellentesque auctor nisi id magna consequat sagittis. Asdf asdfasdfasdfasdf asdfa asdf asdf blah. Curabitur
dapibus enim sit amet elit pharetra tincidunt feugiat nisl imperdiet. There is something impervious going on here. Ut
convallis libero in urna ultrices accumsan. Donec sed odio eros. Donec viverra mi quis quam pulvinar at malesuada arcu
rhoncus. Cum sociis natoque penatibus et magnis dis parturient montes, nascetur ridiculus mus. In rutrum accumsan
ultricies. Mauris vitae nisi at sem facilisis semper ac in est.

\section*{Results}

\subsection*{Kwisatz Haderach}

A beginning is the time for taking the most delicate care that the balances are correct as in \tfig{single}. This every
sister of the Bene Gesserit knows. \comment{Remove `disable' from the todonotes package arguments to display helpful
inline comments like these. Additional commands can be defined for different authors.} To begin your study of the life
of Muad'Dib (\fig{single}{a}), then take care that you first place him in his time: born in the 57th year of the
Padishah Emperor, Shaddam IV. And take the most special care that you locate Muad'Dib in his place: the planet Arrakis.
Do not be deceived by the fact that he was born on Caladan and lived his first fifteen years there. Arrakis, the planet
known as Dune, is forever his place.

\subsection*{Bene Gesserit}

There are equations such as \eqn{skaggs} and other things to be known\cite{Hill78a}, but that is not all. There is in
all things a pattern that is part of our universe. It has symmetry, elegance, and grace -- those qualities you find
always in that which the true artist captures. You can find it in the turning of the seasons, in the way sand trails
along a ridge, in the branch clusters of the creosote bush or the pattern of its leaves. We try to copy these patterns
in our lives and our society, seeking the rhythms, the dances, the forms that comfort. Yet, it is possible to see peril
in the finding of ultimate perfection. It is clear that the ultimate pattern contains it own fixity. In such
perfection, all things move toward death.

 I must not fear. Fear is the mind-killer (\suppfig{template}). Fear is the little-death that brings total
obliteration. I will face my fear. I will permit it to pass over me and through me. And when it has gone past I will
turn the inner eye to see its path. Where the fear has gone there will be nothing. Only I will remain. \note{This is
the litany against fear}

\section*{Discussion}

The present study demonstrated lorem ipsum dolor sit amet, consectetur adipiscing elit. Donec a diam lectus. Sed sit
amet ipsum mauris. Maecenas congue ligula ac quam viverra nec consectetur ante hendrerit. Donec et mollis dolor.
Praesent et diam eget libero egestas mattis sit amet vitae augue. Nam tincidunt congue enim, ut porta lorem lacinia
consectetur. Donec ut libero sed arcu vehicula ultricies a non tortor. Lorem ipsum dolor sit amet, consectetur
adipiscing elit. Aenean ut gravida lorem. Ut turpis felis, pulvinar a semper sed, adipiscing id dolor. Pellentesque
auctor nisi id magna consequat sagittis. Curabitur dapibus enim sit amet elit pharetra tincidunt feugiat nisl
imperdiet. Ut convallis libero in urna ultrices accumsan. Donec sed odio eros. Donec viverra mi quis quam pulvinar at
malesuada arcu rhoncus. Cum sociis natoque penatibus et magnis dis parturient montes, nascetur ridiculus mus. In rutrum
accumsan ultricies. Mauris vitae nisi at sem facilisis semper ac in est.

Lorem ipsum dolor sit amet, consectetur adipiscing elit. Donec a diam lectus. Sed sit amet ipsum mauris. Maecenas
congue ligula ac quam viverra nec consectetur ante hendrerit. Donec et mollis dolor. Praesent et diam eget libero
egestas mattis sit amet vitae augue. Nam tincidunt congue enim, ut porta lorem lacinia consectetur. Donec ut libero sed
arcu vehicula ultricies a non tortor. Lorem ipsum dolor sit amet, consectetur adipiscing elit. Aenean ut gravida lorem.
Ut turpis felis, pulvinar a semper sed, adipiscing id dolor. Pellentesque auctor nisi id magna consequat sagittis.
Curabitur dapibus enim sit amet elit pharetra tincidunt feugiat nisl imperdiet. Ut convallis libero in urna ultrices
accumsan. Donec sed odio eros. Donec viverra mi quis quam pulvinar at malesuada arcu rhoncus. Cum sociis natoque
penatibus et magnis dis parturient montes, nascetur ridiculus mus. In rutrum accumsan ultricies. Mauris vitae nisi at
sem facilisis semper ac in est.

Lorem ipsum dolor sit amet, consectetur adipiscing elit. Donec a diam lectus. Sed sit amet ipsum mauris. Maecenas
congue ligula ac quam viverra nec consectetur ante hendrerit. Donec et mollis dolor. Praesent et diam eget libero
egestas mattis sit amet vitae augue. Nam tincidunt congue enim, ut porta lorem lacinia consectetur. Donec ut libero sed
arcu vehicula ultricies a non tortor. Lorem ipsum dolor sit amet, consectetur adipiscing elit. Aenean ut gravida lorem.
Ut turpis felis, pulvinar a semper sed, adipiscing id dolor.

Pellentesque auctor nisi id magna consequat sagittis. Curabitur dapibus enim sit amet elit pharetra tincidunt feugiat
nisl imperdiet. Ut convallis libero in urna ultrices accumsan. Donec sed odio eros. Donec viverra mi quis quam pulvinar
at malesuada arcu rhoncus. Cum sociis natoque penatibus et magnis dis parturient montes, nascetur ridiculus mus. In
rutrum accumsan ultricies. Mauris vitae nisi at sem facilisis semper ac in est.

    %!TEX root = main.tex

\section*{End Notes}

\paragraph{Acknowledgements} This work was supported by NIH grants lorem ipsum dolor sit amet, consectetur adipiscing
elit. Donec a diam lectus. Sed sit amet ipsum mauris. Maecenas congue ligula ac quam viverra nec consectetur ante
hendrerit. Donec et mollis dolor. Praesent et diam eget libero egestas mattis sit amet vitae augue. Nam tincidunt
congue enim, ut porta lorem lacinia consectetur.

\paragraph{Author Contributions} Donec ut libero sed arcu vehicula ultricies a non tortor. Lorem ipsum dolor sit amet,
consectetur adipiscing elit. Aenean ut gravida lorem. Ut turpis felis, pulvinar a semper sed, adipiscing id dolor.
Pellentesque auctor nisi id magna consequat sagittis. Curabitur dapibus enim sit amet elit pharetra tincidunt feugiat
nisl imperdiet; all authors discussed the results and commented on the manuscript.


    %!TEX root = main.tex

\section*{Figure Legends}

\maincaption{single}{Lorem ipsum dolor sit amet, consectetur adipiscing elit.}{
\panel{A,} Maecenas congue ligula ac quam viverra nec consectetur ante hendrerit. Donec et mollis dolor. Praesent et
diam eget libero egestas mattis sit amet vitae augue. Nam tincidunt congue enim, ut porta lorem lacinia consectetur.
Donec ut libero sed arcu vehicula ultricies a non tortor. \panel{B,} Lorem ipsum dolor sit amet, consectetur adipiscing
elit. Aenean ut gravida lorem. Ut turpis felis, pulvinar a semper sed, adipiscing id dolor. Pellentesque auctor nisi id
magna consequat sagittis. \panel{C,} Curabitur dapibus enim sit amet elit pharetra tincidunt feugiat nisl imperdiet. Ut
convallis libero in urna ultrices accumsan. Donec sed odio eros. \panel{D,} Donec viverra mi quis quam pulvinar at
malesuada arcu rhoncus. Cum sociis natoque penatibus et magnis dis parturient montes, nascetur ridiculus mus. In rutrum
accumsan ultricies. Mauris vitae nisi at sem facilisis semper ac in est. }

\maincaption{double}{Maecenas congue ligula ac quam viverra nec consectetur}{
Ante hendrerit. Donec et mollis dolor. Praesent et diam eget libero egestas mattis sit amet vitae augue. Nam tincidunt
congue enim, ut porta lorem lacinia consectetur. Donec ut libero sed arcu vehicula ultricies a non tortor. Lorem ipsum
dolor sit amet, consectetur adipiscing elit. Aenean ut gravida lorem. Ut turpis felis, pulvinar a semper sed,
adipiscing id dolor. Pellentesque auctor nisi id magna consequat sagittis. Curabitur dapibus enim sit amet elit
pharetra tincidunt feugiat nisl imperdiet. Ut convallis libero in urna ultrices accumsan. Donec sed odio eros. Donec
viverra mi quis quam pulvinar at malesuada arcu rhoncus. Cum sociis natoque penatibus et magnis dis parturient montes,
nascetur ridiculus mus. In rutrum accumsan ultricies. Mauris vitae nisi at sem facilisis semper ac in est. }


    %!TEX root = main.tex

\section*{Online Methods}

\paragraph{Subjects.} 
Adult male wombles were \lipsum[8-9]

\paragraph{Place fields.}
The spatial information content\cite{SkagMcNa93a} of spiking activity was computed as
\begin{equation}
  I=\sum_{i=1}^{N} \; p_i \; \frac{r_i}{R} \; \log_2\left(\frac{r_i}{R}\right)
  \label{eq:skaggs}
\end{equation}
where $p_i$ is the occupancy probability and $r_i$ is the average firing rate of the $i$th bin, and $R$ is the average firing rate for the session. 


  }

  \ifx\excluderefs\undefined
    {
      \setlength{\baselineskip}{1.8em}
      \bibliographystyle{nature}
      \bibliography{./minimal}
    }
  \fi
  
  \ifx\excludefigures\undefined
    {
      \pagestyle{empty}
      \addtolength{\hoffset}{-9mm}
      \addtolength{\textwidth}{18mm}
      %!TEX root = ../main.tex


% Set up the title and figures with captions for the supplementary information file

\begin{center}
  {\Large{\bf Supplementary Information $|$} %!TEX root = main.tex
Lorem Ipsum Dolor sit Amet, New Words Consectetur Adipiscing Elit: Donec a Diam Lectus

% Title ideas:
%

}
  {\large\authorlistplain} 
\end{center}

\subsection*{Supplementary Figures}

% Bring in the figures and captions (don't put the captions in this file)

\let\definesuppcaption\suppfigurewithcaption
%!TEX root = ../main.tex


% Write supplementary figure captions here using this command:
% \definesuppcaption{<figure_name>}{<caption_title>}{<caption_body>}


\definesuppcaption{template}{Illustrations of the widget generating procedure}{
  Lorem ipsum dolor sit amet, consectetur adipiscing elit. Donec a diam lectus. Sed sit amet ipsum mauris. Maecenas congue ligula ac quam viverra nec consectetur ante hendrerit. Donec et mollis dolor. Praesent et diam eget libero egestas mattis sit amet vitae augue. Nam tincidunt congue enim, ut porta lorem lacinia consectetur. Donec ut libero sed arcu vehicula ultricies a non tortor. Lorem ipsum dolor sit amet, consectetur adipiscing elit. Aenean ut gravida lorem. Ut turpis felis, pulvinar a semper sed, adipiscing id dolor. Pellentesque auctor nisi id magna consequat sagittis. 
}




    }
  \fi

  % Include supplementary figure captions at end of main manuscript file for
  % final submission  
  %!TEX root = ../main.tex


% Section title for including the legends at the end of the main manuscript file

\section*{Supplementary Figure Legends}


% Bring in the captions (don't write them in this file)

\let\definesuppcaption\suppcaption
%!TEX root = ../main.tex


% Write supplementary figure captions here using this command:
% \definesuppcaption{<figure_name>}{<caption_title>}{<caption_body>}


\definesuppcaption{template}{Illustrations of the widget generating procedure}{
  Lorem ipsum dolor sit amet, consectetur adipiscing elit. Donec a diam lectus. Sed sit amet ipsum mauris. Maecenas congue ligula ac quam viverra nec consectetur ante hendrerit. Donec et mollis dolor. Praesent et diam eget libero egestas mattis sit amet vitae augue. Nam tincidunt congue enim, ut porta lorem lacinia consectetur. Donec ut libero sed arcu vehicula ultricies a non tortor. Lorem ipsum dolor sit amet, consectetur adipiscing elit. Aenean ut gravida lorem. Ut turpis felis, pulvinar a semper sed, adipiscing id dolor. Pellentesque auctor nisi id magna consequat sagittis. 
}





  % Set up the supplementary info
  \pagebreak
  \setcounter{page}{1}
  \setcounter{figure}{0}  
  %!TEX root = ../main.tex

\begin{titlepage}
  
  \begin{center}
    {\Large{\bf Supplementary Information $|$} %!TEX root = main.tex
Lorem Ipsum Dolor sit Amet, New Words Consectetur Adipiscing Elit: Donec a Diam Lectus

% Title ideas:
%

}
    % { \large \authorlistplain }
  \end{center}
  
  \comment{This page (which should be uploaded as an individual file called \texttt{SIGuide.pdf}) is a list of the
  individual items of Supplementary Information, consisting of the filename, a title, and a brief description of the
  contents of the file.}
  
  \setlength{\baselineskip}{1.3em}
  
  \begin{enumerate}
    \item \correspondinglastname{SI}.pdf
    \begin{description}
      \item \textbf{Supplementary Figures and Legends 1--$N$; Supplementary Table 1}\\[0.1in]
      Supplementary Figures and Legends 1--$N$ and a Supplementary Table with some important numbers.
    \end{description}
  \end{enumerate}
  \pagebreak

\end{titlepage}


  %!TEX root = ../main.tex


% Set up the title and figures with captions for the supplementary information file

\begin{center}
  {\Large{\bf Supplementary Information $|$} %!TEX root = main.tex
Lorem Ipsum Dolor sit Amet, New Words Consectetur Adipiscing Elit: Donec a Diam Lectus

% Title ideas:
%

}
  {\large\authorlistplain} 
\end{center}

\subsection*{Supplementary Figures}

% Bring in the figures and captions (don't put the captions in this file)

\let\definesuppcaption\suppfigurewithcaption
%!TEX root = ../main.tex


% Write supplementary figure captions here using this command:
% \definesuppcaption{<figure_name>}{<caption_title>}{<caption_body>}


\definesuppcaption{template}{Illustrations of the widget generating procedure}{
  Lorem ipsum dolor sit amet, consectetur adipiscing elit. Donec a diam lectus. Sed sit amet ipsum mauris. Maecenas congue ligula ac quam viverra nec consectetur ante hendrerit. Donec et mollis dolor. Praesent et diam eget libero egestas mattis sit amet vitae augue. Nam tincidunt congue enim, ut porta lorem lacinia consectetur. Donec ut libero sed arcu vehicula ultricies a non tortor. Lorem ipsum dolor sit amet, consectetur adipiscing elit. Aenean ut gravida lorem. Ut turpis felis, pulvinar a semper sed, adipiscing id dolor. Pellentesque auctor nisi id magna consequat sagittis. 
}




  %!TEX root = ../main.tex

\subsection*{Supplementary Tables}

{
  \small
  {\bf Supplementary Table 1} $|$ Cell and sample counts by anatomical area. 
}

\begin{center}
\begin{tabular}{clcc}
  \toprule
  {Anatomical Region} & {Aspect}  & {Cells} & {Samples} \\
  \midrule
  \multirow{3}{*}{ABC}
    & Anterior 	&	100	& 400	\\
    & Medial 	  &	50  &	200	\\
    & Posterior &	80	&	320	\\
    \cmidrule{3-4}
    & {} 	      & 230	& 920 \\
	  \cmidrule{2-4}
  \multirow{3}{*}{XYZ}
    & Anterior 	&	60	&	240	\\
    & Medial 	  &	40  &	150	\\
    & Posterior &	50	&	210	\\
    \cmidrule{3-4}
	  & {} 		    & 150	& 600 \\
	  \midrule
  Total & {}    & 380	& 1,520	\\
  \bottomrule
\end{tabular}
\end{center}



\end{document}
