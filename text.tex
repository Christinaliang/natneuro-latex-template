%!TEX root = main.tex

Lorem ipsum dolor sit amet, consectetur adipiscing elit\cite{OKeeNade78a}. \lipsum[2-3]

\section*{Results}

\subsection*{Kwisatz Haderach}

A beginning is the time for taking the most delicate care that the balances are correct as in \tfig{single}. This every
sister of the Bene Gesserit knows. \comment{Remove `disable' from the todonotes package arguments to display helpful
inline comments like these. Additional commands can be defined for different authors.} To begin your study of the life
of Muad'Dib (\fig{single}{a}), then take care that you first place him in his time: born in the 57th year of the
Padishah Emperor, Shaddam IV. And take the most special care that you locate Muad'Dib in his place: the planet Arrakis.
Do not be deceived by the fact that he was born on Caladan and lived his first fifteen years there. Arrakis, the planet
known as Dune, is forever his place.

\subsection*{Bene Gesserit}

There are equations such as \eqn{skaggs} and other things to be known\cite{Hill78a}, but that is not all. There is in
all things a pattern that is part of our universe. It has symmetry, elegance, and grace -- those qualities you find
always in that which the true artist captures. You can find it in the turning of the seasons, in the way sand trails
along a ridge, in the branch clusters of the creosote bush or the pattern of its leaves. We try to copy these patterns
in our lives and our society, seeking the rhythms, the dances, the forms that comfort. Yet, it is possible to see peril
in the finding of ultimate perfection. It is clear that the ultimate pattern contains it own fixity. In such
perfection, all things move toward death.

I must not fear. Fear is the mind-killer (\suppfig{template}). Fear is the little-death that brings total
obliteration. I will face my fear. I will permit it to pass over me and through me. And when it has gone past I will
turn the inner eye to see its path. Where the fear has gone there will be nothing. Only I will remain. \note{This is
the litany against fear}

\section*{Discussion}

\lipsum[4-6]